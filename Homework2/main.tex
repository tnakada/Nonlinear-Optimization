\documentclass{article}



\usepackage{amssymb, amsmath,amsthm,amstext}
\usepackage{tikz,pgfplots}
\usepackage{graphicx}
\newif\ifsolution
\newif\ifsolbox
\setlength{\parskip}{0pt} % 1ex plus 0.5ex minus 0.2ex}
\setlength{\parindent}{0pt}
\usepackage[ letterpaper,left=20mm,right=20mm,bottom=20mm ]{geometry}
\newcommand{\spacer}{\centering{ \begin{minipage}{5mm} \hfill\vspace{8mm} \end{minipage} }}
\usepackage{algorithm,algorithmic}

\usepackage{enumitem}
\setitemize[0]{leftmargin=5pt,itemindent=5pt,itemsep=0mm}

\usepackage{ marvosym }
\usepackage{titling}
\usepackage{fancyhdr}
\usepackage[mmddyyyy,hhmmss]{datetime}
\pagestyle{fancy}
\rfoot{\footnotesize \today\ at \currenttime}
\cfoot{}
\lfoot{}
\rhead{\theauthor}
\lhead{Math 347 - Spring 2020 \thetitle}
\rfoot{\footnotesize \today}

% \usepackage{fullpage}

\newcommand{\R}{\mathbb{R}}
\newcommand{\N}{\mathbb{N}}
\newcommand{\C}{\mathbb{C}}
\newcommand{\diag}{\mathrm{diag}}
\newcommand{\lrvec}[1]{\left[ \begin{array}{r} #1 \end{array}\right]}
\newcommand{\hf}{\frac{1}{2}}
\newtheorem{ex}{Exercise}
\newtheorem{sol}{Solution}

\newcommand{\myVec}[1]{\left( \begin{array}{r} #1 \end{array}\right)}
\newcommand{\myVecT}[1]{\left( \begin{array}{rr} #1 \end{array}\right)^\top}
\newcommand{\myMatTwo}[1]{\left( \begin{array}{rr} #1 \end{array}\right)}


\newcommand{\CF}{\mathcal{F}}
\newcommand{\CM}{\mathcal{M}}
\newcommand{\CE}{\mathcal{E}}
\newcommand{\CI}{\mathcal{I}}
\newcommand{\CC}{\mathcal{C}}
\newcommand{\CA}{\mathcal{A}}
\newcommand{\CO}{\mathcal{O}}

\newcommand{\bfA}{\mathbf{A}}
\newcommand{\bfa}{\mathbf{a}}
\newcommand{\bfB}{\mathbf{B}}
\newcommand{\bfC}{\mathbf{C}}
\newcommand{\bfD}{\mathbf{D}}
\newcommand{\bfd}{\mathbf{d}}
\newcommand{\bfG}{\mathbf{G}}
\newcommand{\bff}{\mathbf{f}}

\newcommand{\bfH}{\mathbf{H}}
\newcommand{\bfI}{\mathbf{I}}
\newcommand{\bfX}{\mathbf{X}}
\newcommand{\bfx}{\mathbf{x}}
\newcommand{\bfb}{\mathbf{b}}
\newcommand{\bfP}{\mathbf{P}}
\newcommand{\bfp}{\mathbf{p}}
\newcommand{\bfM}{\mathbf{M}}
\newcommand{\bfL}{\mathbf{L}}
\newcommand{\bfU}{\mathbf{U}}
\newcommand{\bfu}{\mathbf{u}}
\newcommand{\bfQ}{\mathbf{Q}}
\newcommand{\bfq}{\mathbf{q}}
\newcommand{\bfR}{\mathbf{R}}
\newcommand{\bfT}{\mathbf{T}}
\newcommand{\bfs}{\mathbf{s}}
\newcommand{\bfr}{\mathbf{r}}
\newcommand{\bfv}{\mathbf{v}}
\newcommand{\bfw}{\mathbf{w}}
\newcommand{\bfy}{\mathbf{y}}



\newcommand{\proj}{\mathrm{proj}}
\newcommand{\cond}{\mathrm{cond}}
\newcommand{\rank}{\mathrm{rank}}
\newcommand{\fl}{\mathrm{fl}}



\newdimen\iwidth
\newdimen\iheight

\newcommand{\lrBox}[1]{\fbox{\parbox{\textwidth}{#1}}}
\newcommand{\reg}[1]{{\normalfont{#1}}}
\newcommand{\regtt}[1]{{\normalfont{\texttt{#1}}}}


 \solutionfalse
   % \solutiontrue


\title{\\Homework 2: Unconstrained Optimization}
\author{Taikan Nakada\\ Departments of Math\&CS, Emory University}

\setcounter{ex}{0}
\setcounter{sol}{0}

\begin{document}
	
	This homework is about first-order necessary conditions and the classification of matrices, which were covered in Sections 2.1 and 2.2 of the lecture notes. 
	
	\begin{ex}
			% Beck Exercise 2.17 (i) and (ii). 
		Compute (and explicitly list) all stationary points of
		\begin{equation*}
				f(x) = 4 x_2^3- 12 x_2^2 + 3 x_1^2 x_2.
		\end{equation*}
	\end{ex}
	
	\bigskip
	
	The gradient is 
	\begin{equation*}
	    \nabla f(x) = \left[
	                     \begin{array}{r}
	                        6x_1 x_2 \\
	                        12x_2^2 -24x_2 + 3x_1^2
	                     \end{array}     
	                  \right]
	\end{equation*}
	
    The stationary points need to satisfy $\nabla f(x) = 0$, which is
    \begin{equation*}
        6x_1 x_2 = 0 \quad \text{ and } \quad 
        12x_2^2 -24x_2 + 3x_1^2 = 0.
    \end{equation*}
    
    The first condition gives two cases:
    
    \begin{enumerate}
            \item ${x_1=0 : }$ The second condition gives
            \begin{align*}
                12x_2^2 -24x_2 &= 12\left( x_2^2 -2x_2 \right)\\
                &= x_2\left( x_2 - 1\right) = 0 
                \Rightarrow x_2 = 0,1
            \end{align*}
            \item ${x_2=0 : }$ The second condition gives
            \begin{equation*}
                3 x_1^2 = 0 \Rightarrow x_1 = 1.
            \end{equation*}
        \end{enumerate}
        Thus, the function has stationary points at $(0,0)^\top$ and $(0,1)^\top$.
        
        \bigskip
	
	\begin{ex}[Beck, Problem 2.7 (i)]
		Let $A\in\R^{n\times n}$  be symmetric. Show that $A$ is positive semidefinite if and only if there exists a matrix $B \in \R^{n\times n}$ such that $A = B B^\top$.
	\end{ex}
	
	\bigskip
	
	$\left( \Longrightarrow \right):$ Assume $A \in\R^{n\times n}$ is symmetric positive semi-definite (spsd). By spectral decomposition, there exists an orthogonal matrix $U$ and a diagonal matrix $D = diag \left( \lambda_1, \lambda_2, ..., \lambda_n \right) \in \R^{n\times n}$ such that $A = UDU^\top$, where the columns of $U$ are the corresponding eigenvectors of the values in $D$. 
	
	\medskip
	
	A matrix is spsd if and only if the eigenvalues are non-negative, that is, $\lambda \geq 0$. So we can write $D = \sqrt{D}\sqrt{D}$ where $\sqrt{D} = diag \left( \sqrt{\lambda_1}, \sqrt{\lambda_2}, ..., \sqrt{\lambda_n} \right)$. Then, $A = \left( U\sqrt{D} \right) \left( \sqrt{D}U^\top \right)$ where $U\sqrt{D} = B$ and $\sqrt{D}U^\top = B^\top.$ 
	
	\medskip
	
	Therefore, $A = BB^\top.$
	
	\bigskip
	
	$\left( \Longleftarrow \right):$ Let $B \in\R^{n\times n}$ and $A = BB^\top$. Then, $\left( \vec {v}^\top B \right) \left( B^\top \vec{v}  \right) = \vec{x}^\top \vec{x} \geq 0$. It follows that $\vec{x}^\top A \vec{x} \geq 0,$ $\forall x \in \R^n$, thus $A$ is spsd. $\qed$
	
    \bigskip

	\begin{ex}[Beck, Problem 2.7 (ii)]
		Let $x\in \R^n$ and let $A\in\R^{n\times n}$  be defined as
			\begin{equation*}
				A_{ij} = x_i x_j, \quad i,j=1,2,\ldots,n.
			\end{equation*}
		\begin{enumerate}
			\item Show that $A$ is positive semidefinite.
			    
			    \bigskip
			    
	            $A \in \R^{n \times n}$ is defined as $A_{ij} = x_i x_j, \quad i,j=1,2,\ldots,n.$ This can be written as 
			    \begin{equation*} A = 
			        \begin{pmatrix}
			            x_{11} & x_{12} & \cdots & x_{1n} \\
			            x_{21} & x_{22} & \cdots & x_{2n} \\
			            \vdots & \vdots & & \vdots \\
			            x_{n1} & x_{n2} & \cdots & x_{nn} \\
			        \end{pmatrix}.
			    \end{equation*}
			    Now, $x_i x_j = x_j x_i$ by the commutative property. Since $A$ is a square matrix and $A = x_i x_j = x_j x_i = A\top$, it is a symmetric matrix. By spectral decomposition any symmetric matrix can be diagonalized, so A can be diagonalized as $U^\top AU = D$. This also leads us to the fact that the eigenvalues of $A$ are always non-negative. By the Eigenvalue Characterization Theorem, $A$ is positive semidefinite iff all its eigenvalues are non-negative. Therefore, $A$ is spsd.
			
			\item Show that $A$ is not positive definite for any $x$ when $n>1$.
			
			    \bigskip
			
			    Assume to the contrary  that $A$ is a positive definite matrix. Let $D_k (A)$ be the determinant of the upper left $k \times k$ submatrix of $A$. By the Principle Minors Criterion, A must satisfy $D_1 (A) > 0, D_2 (A) > 0, ..., D_n (A) > 0$. However, every $A \in \R^{n \times n}$ where $n > 1$ has the submatrix 
			    \begin{equation*}
			        \left(
			        \begin{array}{rr}
			            1 & 2\\
			            2 & 4\\
			        \end{array}
			        \right),
			    \end{equation*}
			    where $D_2 (A) = 0.$ \Lightning This is a contradiction, and $A$ is therefore not positive definite for any $x$ when $n>1$.
		\end{enumerate}	
	\end{ex}
	
    
	
	\begin{ex}[Beck, Problem 2.11]
	Let $d \in \Delta_n$ ($\Delta_n$ being the unit-simplex). Show that the $n\times n$ matrix $A$ defined by 
	$$
	A_{ij} = \begin{cases}
	d_i - d_i^2, & i = j, \\
	-d_i d_j, & i \neq j,
	\end{cases}
	$$
	is positive semidefinite.
	
	\emph{Hint:} Use Theorem 2.5 from the notes.
	\end{ex}
	
	Define unit simplex as 
	\begin{equation*}
	   \Delta_n = \left\{ x \in \R^n_{+} \ : \ \quad \sum_{i=1}^n x_i = 1 \right\}.
	\end{equation*}
	Then, the off diagonal elements $-d_i d_j$ for $i \neq j$ are $\leq 0$. For the case of $i = j$, $d_i - d_i^2$ is always $\geq 0$ since $d_i \leq 1$. 
	
	Let 
	\begin{equation*}
	    d = \left\{d_i, d_j\right\} = \left\{ \frac{1}{n} , \frac{n-1}{n} \right\} \in \Delta_n
	\end{equation*} be the set of possible elements of $d$. Then, for the diagonal elements when $i = j$ we have that 
	\begin{align*}
	    d_i - d_i^2 &= \frac{1}{n} - \left(\frac{1}{n}\right)^2 \\
	    &= \frac{1}{n} - \frac{1}{n^2} \\
	    &= \frac{n-1}{n},
	\end{align*}
    and the non-diagonal elements for $i \neq j$ are 
    \begin{align*}
        -d_i d_j &= -\left(\frac{n-1}{n}\right) \left(\frac{1}{n}\right) \\
        &= -\frac{n-1}{n^2}. 
    \end{align*}
    
    This holds for an arbitrary $n+1$ since 
	\begin{align*}
	    d_i - d_i^2 &= \frac{1}{n+1} - \left(\frac{1}{n+1}\right)^2 \\
	    &= \frac{1}{n+1} - \frac{1}{\left(n+1\right)^2} \\
	    &= \frac{\left( n+1\right) - 1}{\left(n+1\right)^2} \\
	    &= \frac{n}{\left(n+1\right)^2}
	\end{align*}
	
    and the off-diagonals are  
    \begin{align*}
        -d_i d_j &= -\left(\frac{\left(n+1\right)-1}{n+1}\right) \left(\frac{1}{n+1}\right) \\
        &= -\frac{n}{\left(n+1\right)^2}.
    \end{align*}
    
    This shows that $A$ is diagonally dominant since $|A_{i=j}|\geq\sum_{j\neq i} |A_{ij}|$. We have already proved that the diagonal entries are nonnegative. Thus, by Theorem 2.5 $A$ is positive semidefinite.$\qed$ 
	\end{document}
	