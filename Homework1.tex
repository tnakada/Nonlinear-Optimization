\documentclass{article}

\usepackage[english]{babel}
\usepackage[utf8]{inputenc}
\usepackage{fancyhdr}
\usepackage[margin=1in]{geometry}
\usepackage{mathtools}
\usepackage{amsfonts}
\newcommand{\R}{\mathbb{R}}

\addtolength{\topmargin}{+.5in}
\pagestyle{fancy}
\fancyhf{}
\rhead{Taikan Nakada}
\lhead{Math 347 - Spring 2020 - Homework 1}

%%%Header%%%

\begin{document}

\noindent This homework is about linear algebra preliminaries and norms, which were covered in Section 1.2 of the lecture notes

\section*{Exercise 1:}

Let
\begin{equation*}
    A = \left(
        \begin{array}{rrrr}
            1 & 1 & -1 & -1 \\
            1 & 1 & 1 & 1 \\
            0 & 1 & 1 & 0 
        \end{array}
    \right), \quad \text { and } \quad
    b = \left(
        \begin{array}{r}
            -1\\
            4\\
            -1\\
            2\\
        \end{array}
        \right)
\end{equation*}
\begin{enumerate}
    \item Compute a basis of the null space of $A$ and a basis of the range space of $A^\top$.
    
    %%%Answer%%%
    \bigskip
    Reduced row echelon form of $A$:

    \begin{equation*}
        A = \left(
            \begin{array}{rrrr}
                1 & 1 & -1 & -1 \\
                1 & 1 & 1 & 1 \\
                0 & 1 & 1 & 0 
            \end{array}
        \right)
        \overset{R_2-R_1}{\longrightarrow}
        \left(
            \begin{array}{rrrr}
                1 & 1 & -1 & -1 \\
                0 & 0 & 2 & 2 \\
                0 & 1 & 1 & 0 
            \end{array}
        \right)
        \overset{\frac{1}{2} R_2 \leftrightarrow R_3}{\longrightarrow}
        \left(
            \begin{array}{rrrr}
                1 & 1 & -1 & -1 \\
                0 & 1 & 1 & 0 \\
                0 & 0 & 2 & 2 
            \end{array}
        \right)
    \end{equation*}
    
    \begin{equation*}
        \overset{R_2-R_3}{\longrightarrow}
        \left(
            \begin{array}{rrrr}
                1 & 1 & -1 & -1 \\
                0 & 1 & 0 & -1 \\
                0 & 0 & 1 & 1 
            \end{array}
        \right)
        \overset{R_1+R_3}{\longrightarrow}
        \left(
            \begin{array}{rrrr}
                1 & 1 & 0 & 0 \\
                0 & 1 & 0 & -1 \\
                0 & 0 & 1 & 1 
            \end{array}
        \right)
        \overset{R_1-R_2}{\longrightarrow}
        \left(
            \begin{array}{rrrr}
                1 & 0 & 0 & 1 \\
                0 & 1 & 0 & -1 \\
                0 & 0 & 1 & 1 
            \end{array}
        \right).
    \end{equation*}
    
    
    
    We want to find a vector $\vec{x}$ that satisfies $A\vec{x}=0:$
    
    \begin{equation*}
        \vec{x} = \left[
                    \begin{array}{r}
                        x_1\\
                        x_2\\
                        x_3\\
                        x_4
                    \end{array}
                  \right]
                = \left[
                    \begin{array}{r}
                        x_4\\
                        -x_4\\
                        x_4\\
                        x_4
                    \end{array}
                  \right]
                = x_4 \left[
                        \begin{array}{r}
                            1\\
                            -1\\
                            1\\
                            1
                        \end{array}
                      \right]
    \end{equation*}
    Therefore, the null space of A is
    
    \begin{equation*}
        NullSpace\left(A\right) = \left\{ \vec{x} \in \R^4 \ : \vec{x} = \left[
            \begin{array}{r}
                1\\
                -1\\
                1\\
                1
            \end{array}
        \right], \quad \forall x_4 \in \R^4 
        \right\} = Span\left[
                            \begin{array}{r}
                                1\\
                                -1\\
                                1\\
                                1
                            \end{array}
                        \right]
    \end{equation*}
    
    Now, the reduced row echelon form of $A^\top$ is 
    
    \begin{equation*}
    A^\top = \left(
            \begin{array}{rrr}
                1 & 1 & 0\\
                1 & 1 & 1\\
                -1 & 1 & 1\\
                -1 & 1 & 0
            \end{array}
        \right)
        \overset{R_2-R_1}{\longrightarrow}
        \left(
            \begin{array}{rrr}
                1 & 1 & 0\\
                0 & 0 & 1\\
                -1 & 1 & 1\\
                -1 & 1 & 0
            \end{array}
        \right)
        \overset{R_3+R_1}{\underset{R_4+R_1}\longrightarrow}
        \left(
            \begin{array}{rrr}
                1 & 1 & 0\\
                0 & 0 & 1\\
                0 & 2 & 1\\
                0 & 2 & 0
            \end{array}
        \right)
        \overset{\frac{1}{2} R_3 \leftrightarrow R_2}{\longrightarrow}
    \end{equation*}
    \begin{equation*}
        \left(
            \begin{array}{rrr}
                1 & 1 & 0\\
                0 & 1 & \frac{1}{2}\\
                0 & 0 & 1\\
                0 & 2 & 0
            \end{array}
        \right)
        \overset{R_4-2R_2}{\longrightarrow}
        \left(
            \begin{array}{rrr}
                1 & 1 & 0\\
                0 & 1 & \frac{1}{2}\\
                0 & 0 & 1\\
                0 & 0 & -1
            \end{array}
        \right)
        \overset{R_4+R_3}{\longrightarrow}
        \left(
            \begin{array}{rrr}
                1 & 1 & 0\\
                0 & 1 & \frac{1}{2}\\
                0 & 0 & 1\\
                0 & 0 & 0
            \end{array}
        \right)
        \overset{R_2-\frac{1}{2}R_3}{\underset{R_1-R_2}\longrightarrow}
        \left(
            \begin{array}{rrr}
                1 & 0 & 0\\
                0 & 1 & 0\\
                0 & 0 & 1\\
                0 & 0 & 0
            \end{array}
        \right)
    \end{equation*}
    

    %%%End Answer%%%
    
    \item Write $b = p + q$ where $p$ is in the null space of $A$ and $Q$ in the range space of $A^\top$.
\end{enumerate}

\section*{Exercise 2:}

Show that $\| \cdot \|_p$ for $p=1/3$ is not a norm 

\section*{Exercise 3:}

Show that for all $x\in\R^n$, $\|x\|_{\infty} \leq \|x\|_2 \leq \sqrt{n}\|x\|_{\infty}$.

\section*{Exercise 4:}

What are the relations between $\|A \|$ and $\|A^\top\|$ for the matrix norms $\|\cdot\|_1, \|\cdot\|_2, \|\cdot\|_{\infty}$? Describe your ideas in a few sentences and consider adding an example or sketch of a proof.

\end{document}
