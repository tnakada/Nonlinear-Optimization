\documentclass{article}

\usepackage[english]{babel}
\usepackage[utf8]{inputenc}
\usepackage{fancyhdr}
\usepackage[margin=1in]{geometry}
\usepackage{mathtools}
\usepackage{amsfonts}
\usepackage{ stmaryrd }
\newcommand{\R}{\mathbb{R}}
\pagestyle{fancy}
\fancyhf{}
\rhead{Taikan Nakada}
\lhead{Math 347 - Spring 2020 - Homework 1}

%%%Header%%%

\begin{document}

\noindent This homework is about linear algebra preliminaries and norms, which were covered in Section 1.2 of the lecture notes

\section*{Exercise 1:}

Let
\begin{equation*}
    A = \left(
        \begin{array}{rrrr}
            1 & 1 & -1 & -1 \\
            1 & 1 & 1 & 1 \\
            0 & 1 & 1 & 0 
        \end{array}
    \right), \quad \text { and } \quad
    b = \left(
        \begin{array}{r}
            -1\\
            4\\
            -1\\
            2\\
        \end{array}
        \right)
\end{equation*}
\begin{enumerate}
    \item Compute a basis of the null space of $A$ and a basis of the range space of $A^\top$.
    
    %%%Answer%%%
    \bigskip
    Reduced row echelon form of $A$:

    \begin{equation*}
        A = \left(
            \begin{array}{rrrr}
                1 & 1 & -1 & -1 \\
                1 & 1 & 1 & 1 \\
                0 & 1 & 1 & 0 
            \end{array}
        \right)
        \overset{R_2-R_1}{\longrightarrow}
        \left(
            \begin{array}{rrrr}
                1 & 1 & -1 & -1 \\
                0 & 0 & 2 & 2 \\
                0 & 1 & 1 & 0 
            \end{array}
        \right)
        \overset{\frac{1}{2} R_2 \leftrightarrow R_3}{\longrightarrow}
        \left(
            \begin{array}{rrrr}
                1 & 1 & -1 & -1 \\
                0 & 1 & 1 & 0 \\
                0 & 0 & 2 & 2 
            \end{array}
        \right)
    \end{equation*}
    
    \begin{equation*}
        \overset{R_2-R_3}{\longrightarrow}
        \left(
            \begin{array}{rrrr}
                1 & 1 & -1 & -1 \\
                0 & 1 & 0 & -1 \\
                0 & 0 & 1 & 1 
            \end{array}
        \right)
        \overset{R_1+R_3}{\longrightarrow}
        \left(
            \begin{array}{rrrr}
                1 & 1 & 0 & 0 \\
                0 & 1 & 0 & -1 \\
                0 & 0 & 1 & 1 
            \end{array}
        \right)
        \overset{R_1-R_2}{\longrightarrow}
        \left(
            \begin{array}{rrrr}
                1 & 0 & 0 & 1 \\
                0 & 1 & 0 & -1 \\
                0 & 0 & 1 & 1 
            \end{array}
        \right).
    \end{equation*}
    
    
    
    We want to find a vector $\vec{x}$ that satisfies $A\vec{x}=0:$
    
    \begin{equation*}
        \left(
            \begin{array}{rrrr}
                1 & 0 & 0 & 1 \\
                0 & 1 & 0 & -1 \\
                0 & 0 & 1 & 1 
            \end{array}
        \right)\vec{x}=0 \Longrightarrow  
        \left\{
            \begin{array}{r}
                x_1+x_4=0\\
                x_2-x_4=0\\
                x_3+x_4=0
            \end{array}
        \right\}
        \Longrightarrow 
        x_4 \left[
                \begin{array}{r}
                    -1\\
                    1\\
                    -1\\
                    1
                \end{array}
            \right]
    \end{equation*}
    
    Therefore, the null space of A is
    
    \begin{equation*}
        NullSpace\left(A\right) = \left\{ \vec{x} \in \R^4 \ : \vec{x} = \left[
            \begin{array}{r}
                -1\\
                1\\
                -1\\
                1
            \end{array}
        \right], \quad \forall x_4 \in \R^4 
        \right\} = Span\left[
                            \begin{array}{r}
                                -1\\
                                1\\
                                -1\\
                                1
                            \end{array}
                        \right]
    \end{equation*}
    
    Now, the reduced row echelon form of $A^\top$ is 
    
    \begin{equation*}
    A^\top = \left(
            \begin{array}{rrr}
                1 & 1 & 0\\
                1 & 1 & 1\\
                -1 & 1 & 1\\
                -1 & 1 & 0
            \end{array}
        \right)
        \overset{R_2-R_1}{\longrightarrow}
        \left(
            \begin{array}{rrr}
                1 & 1 & 0\\
                0 & 0 & 1\\
                -1 & 1 & 1\\
                -1 & 1 & 0
            \end{array}
        \right)
        \overset{R_3+R_1}{\underset{R_4+R_1}\longrightarrow}
        \left(
            \begin{array}{rrr}
                1 & 1 & 0\\
                0 & 0 & 1\\
                0 & 2 & 1\\
                0 & 2 & 0
            \end{array}
        \right)
        \overset{\frac{1}{2} R_3 \leftrightarrow R_2}{\longrightarrow}
    \end{equation*}
    \begin{equation*}
        \left(
            \begin{array}{rrr}
                1 & 1 & 0\\
                0 & 1 & \frac{1}{2}\\
                0 & 0 & 1\\
                0 & 2 & 0
            \end{array}
        \right)
        \overset{R_4-2R_2}{\longrightarrow}
        \left(
            \begin{array}{rrr}
                1 & 1 & 0\\
                0 & 1 & \frac{1}{2}\\
                0 & 0 & 1\\
                0 & 0 & -1
            \end{array}
        \right)
        \overset{R_4+R_3}{\longrightarrow}
        \left(
            \begin{array}{rrr}
                1 & 1 & 0\\
                0 & 1 & \frac{1}{2}\\
                0 & 0 & 1\\
                0 & 0 & 0
            \end{array}
        \right)
        \overset{R_2-\frac{1}{2}R_3}{\underset{R_1-R_2}\longrightarrow}
        \left(
            \begin{array}{rrr}
                1 & 0 & 0\\
                0 & 1 & 0\\
                0 & 0 & 1\\
                0 & 0 & 0
            \end{array}
        \right)
    \end{equation*}
    
    and the range space of $A^\top$ is
    
    \begin{equation*}
        RangeSpace\left(A^\top\right) = 
        Span\left\{
            \left[
                \begin{array}{r}
                    1\\
                    1\\
                    -1\\
                    -1
                \end{array}
            \right],
            \left[
                \begin{array}{r}
                    1\\
                    1\\
                    1\\
                    1
                \end{array}
            \right],
            \left[
                \begin{array}{r}
                    0\\
                    1\\
                    1\\
                    0
                \end{array}
            \right]
            \right\}
    \end{equation*}

    %%%End Answer%%%
    
    \item Write $b = p + q$ where $p$ is in the null space of $A$ and $Q$ in the range space of $A^\top$.
    
    \begin{equation*}
        b = p + q \Longrightarrow
        \left(
            \begin{array}{r}
                -1\\
                4\\
                -1\\
                2\\
            \end{array}
        \right) = 
        \left(
            \begin{array}{r}
                -1\\
                1\\
                -1\\
                1\\
            \end{array}
        \right) + 3
        \left(
            \begin{array}{r}
                0\\
                1\\
                0\\
                \frac{1}{3}\\
            \end{array}
        \right)
    \end{equation*}
    
\end{enumerate}

\section*{Exercise 2:}

Show that $\| \cdot \|_p$ for $p=1/3$ is not a norm 

%%%Answer%%%

\bigskip
Assume to the contrary that $\| \cdot \|_p$ for $p=1/3$ on a vector space $V$ : $V \longrightarrow \R^n$, defined as 
\begin{equation*}
\| \vec{x}\|_\frac{1}{3} = \left( \sum_{i=1}^n |x_i|^\frac{1}{3} \right)^{3},
\end{equation*}
is a norm. Then, it must satisfy nonnegativity, positive homogeneity, and the triangle inequality. 
Let $\vec{x}, \vec{y} \in V$ where
\begin{equation*}
    \vec{x}=\left(
                \begin{array}{r}
                    1\\0\\0
                \end{array}
            \right) \quad \text { and } \quad
    \vec{y}=\left(
                \begin{array}{r}
                    0\\1\\0
                \end{array}
            \right) \quad \text { where } \quad
    \vec{x}+\vec{y}=\left(
                \begin{array}{r}
                    1\\1\\0
                \end{array}
            \right).
\end{equation*}
Then, by the triangle inequality

\begin{equation}
    \| \vec{x}+\vec{y}\|_\frac{1}{3} \leq \| \vec{x}\|_\frac{1}{3} + 
    \| \vec{y}\|_\frac{1}{3}
\end{equation}
\begin{equation}
    \left( \sum_{i=1}^n |x_i+y_i|^\frac{1}{3}\right)^{3} \leq 
    \left( \sum_{i=1}^n |x_i|^\frac{1}{3}\right)^{3} +
    \left( \sum_{i=1}^n |y_i|^\frac{1}{3}\right)^{3}
\end{equation}
\begin{equation}
    \left(
        1^\frac{1}{3} + 1^\frac{1}{3} + 0^\frac{1}{3}
    \right)^3 \leq
    \left(
        1^\frac{1}{3} + 0^\frac{1}{3} + 0^\frac{1}{3}
    \right)^3 +
    \left(
        0^\frac{1}{3} + 1^\frac{1}{3} + 0^\frac{1}{3}
    \right)^3
\end{equation}
\begin{equation}
    2^3 \leq 1^3 + 1^3
\end{equation}
\begin{equation}
    8 \leq 2 \quad \lightning
\end{equation}
This is a contradiction, and thus the triangle inequality is not satisfied. Therefore, $\| \cdot \|_p$ for $p=1/3$ is not a norm.
%%%End Answer%%%

\section*{Exercise 3:}

Show that for all $x\in\R^n$, $\|x\|_{\infty} \leq \|x\|_2 \leq \sqrt{n}\|x\|_{\infty}$.

%%%Answer%%%
\bigskip

Let $\vec{x} \in \R^n$ where $\vec{x} = \left(x_1, x_2, ... , x_n \right)$. Define the 2-norm and $\infty$-norm as
\begin{equation*}
    \| \vec{x} \|_{\infty} = \max_{i=1,2,\ldots,n} |x_i| \quad \text { and } \quad 
    \| \vec{x}\|_2 = \left( \sum_{i=1}^n |x_i|^2 \right)^{\frac{1}{2}}.
\end{equation*}
We begin by first proving the inequality on the left hand side. 
\begin{equation}
    \| \vec{x} \|_{\infty}^2 = \left(\max_{i=1,2,\ldots,n} |x_i|\right)^2 = 
    \max_{i=1,2,\ldots,n} |x_i|^2 \leq 
    \sum_{i=1}^n |x_i|^2 =
    \| \vec{x}\|_2^2 \quad \Longrightarrow \quad
    \|x\|_{\infty} \leq \|x\|_2
\end{equation}
Now we can take on the right hand side.
\begin{equation}
    \| \vec{x}\|_2 = 
    \sqrt{\sum_{i=1}^n |x_i|^2} \leq 
    \sqrt{\sum_{i=1}^n \| \vec{x} \|_{\infty}^2} =
    \sqrt{n\| \vec{x} \|_{\infty}^2} =
    \sqrt{n}\| \vec{x} \|_{\infty} \quad \Longrightarrow \quad
    \|x\|_2 \leq \sqrt{n}\|x\|_{\infty}
\end{equation}
%%%End Answer%%%

\section*{Exercise 4:}

What are the relations between $\|A \|$ and $\|A^\top\|$ for the matrix norms $\|\cdot\|_1, \|\cdot\|_2, \|\cdot\|_{\infty}$? Describe your ideas in a few sentences and consider adding an example or sketch of a proof.

%%%Answer%%%
\bigskip
Let us begin by defining each norm for matrix $A \in \R^{mxn}$. $\|\cdot\|_1, \|\cdot\|_2$, and  $\|\cdot\|_{\infty}$ are defined as

\begin{equation*}
    \| A \|_1 = \max_{j=1,2,\ldots,n} \sum_{i=1}^m |A_{i,j}|,
\end{equation*}
\begin{equation*}
    \|A \|_2 = \sqrt{\lambda_{\max} (A^\top A)} = \sigma_{\max}(A),
\end{equation*}
where $\lambda_{\max}$ is the largest eigenvalue and $\sigma_{\max}$ the largest singular value, and
\begin{equation*}
    \| A \|_{\infty} = \max_{i=1,2,\ldots,m} \sum_{j=1}^n |A_{i,j}|.
\end{equation*}
$\|A \|$ and $\|A^\top\|$ have the same spectral norm because $\|\cdot\|_2$ gives the largest singular value in the matrix. Taking the transpose of the matrix $A$ does not change the values. 

\vspace{5mm}

\noindent Now, conceptually the 1-norm sums every component of each column, that is, it sums all $m$ components in the $j^{th}$ column. Out of the $n$ columns, the norm returns the sum of the column with the greatest sum.

\vspace{5mm}

\noindent The line of attack for the $\infty$-norm is similar, but essentially "in reverse". It sums all $n$ components in the $i^{th}$ row. Out of the $m$ rows, the norm returns the sum of the row with the greatest sum.

\vspace{5mm}

\noindent With the algorithm in mind, it's straightforward to see that 
\begin{equation*}
    \| A \|_1 = \| A^\top \|_{\infty} \quad \text { and }\quad 
    \| A^\top \|_1 = \| A \|_{\infty}.
\end{equation*}

%%%End Answer%%%

\end{document}
